\begin{apendicesenv}

\partapendices

\chapter{Primeiro Apêndice}

\section{Termo de Abertura do Projeto}
Esse termo foi preparado considerando o exemplo do \cite{lappis_tap}.

\subsection{Descrição e Justificativa do projeto}

A solução proposta para o problema da dificuldade dos cadeirantes em fazer suas compras no supermercado, consiste em construir um carrinho de compras que o seguirá.

\par Dessa forma, esse carrinho deverá agir de forma autônoma e desviar de obstáculos em um ambiente ideal, ou seja, sem rampas. E assim facilitar e tornar mais ágil o momento de compras, de forma a oferecer uma certa independência ao cadeirante.

\par A solução irá cobrir conceitos de cada engenharia que participa do projeto, que trocarão conhecimentos para integrar suas partes no produto final. 

\subsection{Objetivos do projeto}	

\begin{itemize}  
\item Construir um produto que atinja partes de Engenharia Eletrônica, de Software, Automotiva, de Energia e Aeroespacial;
\item Desenvolver protótipos de cada subsistema já definidos, são eles: Automação e Controle, Estrutura e Alimentação;
\item Integrar subsistemas de forma a fazê-los funcionar juntos;
\end{itemize}

\subsection{Escopo (e não escopo) do projeto}
%tabela
% ######## init table ########
\begin{table}[h]
 \centering
% distancia entre a linha e o texto
 {\renewcommand\arraystretch{1.25}
 \caption{Escopo.}
 \begin{tabular}{ l l }
  \cline{1-1}\cline{2-2}  
    \multicolumn{1}{|c|}{\textbf{É escopo} \centering } &
    \multicolumn{1}{c|}{\textbf{Não é escopo} \centering }
  \\  
  \cline{1-1}\cline{2-2}  
    \multicolumn{1}{|p{3.850cm}|}{\begin{center}Que o carrinho seja autônomo
\end{center}  			


\begin{center}Que o carrinho desvie de obstáculos em curvas
\end{center}} &
    \multicolumn{1}{p{4.217cm}|}{\begin{center}Que o carrinho funcione em rampas
\end{center}  			


\begin{center}Que o cadeirante controle o carrinho manualmente
\end{center}}
  \\  
  \hline

 \end{tabular} }
\end{table}

\subsection{Stakeholders e Expectativas}

Os envolvidos nesse projeto são então os alunos, o cadeirante e o mercado. As responsabilidades dos alunos será então o desenvolvimento do projeto, do mercado de fazer a compra do produto e do cadeirante de usar o produto.

\par Pensando na necessidade de um produto de fácil uso e acesso que atenda necessidades de clientes especiais, o mercado espera então obter um diferencial e que seja favorável em questão de custo-benefício para sua empresa.

\par Já a expectativa do cadeirante é de certa forma se incluir e ter facilidade, acessibilidade e agilidade em suas compras, atendendo suas necessidades.

\subsection{Cronograma macro dos marcos do projeto}

Estre cronograma irá evidenciar as principais datas do projeto, que são os Pontos de Controle.
\newline
\par \textbf{- Marco 1:} 
Ponto de controle 1 – 02/09/2016 - Entrega dos planos de gerenciamento e possíveis escolhas de soluções de cada área.
\newline
\par \textbf{- Marco 2:} 
Ponto de controle 2 – 09/11/2016 - Entrega do protótipo de cada subsistema funcionando de forma separada.
\newline
\par \textbf{- Marco 3:}
Ponto de controle 3 – 02/12/2016 - Entrega do protótipo final, ou seja, integração dos subsistemas.

\subsection{Premissas e Restrições organizacionais}

O carrinho deverá de fato ser autônomo, com uma estrutura que suporte um máximo de 20kg de compras, andando em uma velocidade de até 5km/h a partir de um motor elétrico DC. 

\par Dessa forma, os subsistemas dispõem de um conjunto de opções de processadores, sensores, algoritmos, materiais, motores e baterias para realizar essas tarefas. Essas escolhas devem ser feitas de forma sábia, visando um desenvolvimento eficiente do produto.

\newpage
\subsection{Orçamento preliminar}

%tabela
% ######## init table ########
\begin{table}[h]
 \centering
% distancia entre a linha e o texto
 {\renewcommand\arraystretch{1.25}
 \begin{tabular}{ l l l l }
  \cline{1-1}\cline{2-2}\cline{3-3}\cline{4-4}  
    \multicolumn{1}{|p{2.367cm}|}{\begin{center}\textbf{Recurso}
\end{center}} &
    \multicolumn{1}{p{2.367cm}|}{\begin{center}\textbf{Valor estimado}
\end{center}} &
    \multicolumn{1}{p{2.367cm}|}{\begin{center}\textbf{Quantidade}
\end{center}} &
    \multicolumn{1}{p{2.367cm}|}{\begin{center}\textbf{Duração}
\end{center}}
  \\  
  \cline{1-1}\cline{2-2}\cline{3-3}\cline{4-4}  
    \multicolumn{1}{|p{2.367cm}|}{\begin{center}Engenheiro
\end{center}} &
    \multicolumn{1}{p{2.367cm}|}{\begin{center}R\$ 16,50 por hora
\end{center}} &
    \multicolumn{1}{p{2.367cm}|}{\begin{center}13
\end{center}} &
    \multicolumn{1}{p{2.367cm}|}{\begin{center}90h
\end{center}}
  \\  
  \cline{1-1}\cline{2-2}\cline{3-3}\cline{4-4}  
    \multicolumn{1}{|p{2.367cm}|}{\begin{center}Subsistema Controle
\end{center}} &
    \multicolumn{1}{p{2.367cm}|}{\begin{center}R\$ 665,00
\end{center}} &
    \multicolumn{1}{p{2.367cm}|}{\begin{center}1
\end{center}} &
    \multicolumn{1}{p{2.367cm}|}{\begin{center}90h
\end{center}}
  \\  
  \cline{1-1}\cline{2-2}\cline{3-3}\cline{4-4}  
    \multicolumn{1}{|p{2.367cm}|}{\begin{center}Subsistema Estrutura
\end{center}} &
    \multicolumn{1}{p{2.367cm}|}{\begin{center}R\$ 300,00
\end{center}} &
    \multicolumn{1}{p{2.367cm}|}{\begin{center}1
\end{center}} &
    \multicolumn{1}{p{2.367cm}|}{\begin{center}90h
\end{center}}
  \\  
  \cline{1-1}\cline{2-2}\cline{3-3}\cline{4-4}  
    \multicolumn{1}{|p{2.367cm}|}{\begin{center}Subsistema Alimentação
\end{center}} &
    \multicolumn{1}{p{2.367cm}|}{\begin{center}R\$ 800,00
\end{center}} &
    \multicolumn{1}{p{2.367cm}|}{\begin{center}1
\end{center}} &
    \multicolumn{1}{p{2.367cm}|}{\begin{center}90h
\end{center}}
  \\  
  \cline{1-1}\cline{2-2}\cline{3-3}\cline{4-4}  
    \multicolumn{1}{|p{2.367cm}|}{\begin{center}Total
\end{center}} &
    \multicolumn{3}{p{7.100cm}|}{\begin{center}R\$ 3.250,00
\end{center}}
  \\  
  \hline

 \end{tabular} }
\end{table}

\subsection{Restrições e Riscos do projeto}

O projeto se restringe em um produto capaz de seguir um cadeirante de forma autônoma em ambiente ideal. Os riscos estimados do projeto são todos os erros que podem acontecer durante seu desenvolvimento e  estão detalhados no Plano de gerenciamento de Riscos, podendo ser organizacionais, humano, técnico e de projeto.

\par Esses riscos estão relacionados a gestão, aos integrantes e aos subsistemas do projeto.

\subsection{Descrição dos subprodutos}
O produto final será composto de 3 subprodutos, um de cada subsistema. Primeiramente, a parte de hardware, sensoriamento e algoritmos, que fazem parte do controle e automação. Uma estrutura com rodas e uma transmissão, do subsistema Estrutura. E um motor elétrico e baterias, da parte de Alimentação.

\subsection{Plano de gerenciamento}
O gerenciamento do projeto será feito através de planos de gerenciamento de tempo, de riscos, de recursos humanos, de custos, de tempo e de requisitos., que serão executados pelos integrantes do grupo em conjunto.

\subsection{Milestones identificados}
Os milestones identificados são definidos pelos marcos do projeto, e são eles:

\begin{itemize}
\item Proposta e aprovação do projeto;
\item Apresentação de Escopo e Requisitos do projeto;
\item Construção de protótipos para subsistemas definidos;
\item Integração de subsistemas;
\item Entrega do Produto final.
\end{itemize}


\end{apendicesenv}

\chapter[Considerações Finais]{Considerações Finais}
Algumas atividades comuns para as pessoas podem trazer grandes transtornos para cadeirantes. Uma delas é fazer compras em supermercado, por exemplo, pois o cadeirante precisa das mãos para se locomover e movimentar o carrinho \cite{blog_cadeir}. 

\par Já existe no mercado um modelo de carrinho de compras adaptáveis para cadeirante. Eles são acoplados a cadeira de rodas. Porém, pode não ser tão eficiente, além de aumentar o esforço gerado pelo cadeirante devido ao peso adicional tem uma capacidade de 65 litros \cite{blog_defic}. 

\par Esse carrinho irá diminuir o esforço gerado pelo cadeirante, pois será autônomo além de ter uma capacidade volumétrica de 100 litros.

\par Ao final do projeto pretende-se apresentar um protótipo do carrinho mostrando a viabilidade de desenvolver esse carrinho para supermercado e contribuir para uma maior inclusão dos deficientes físicos aos diversos tipos de atividades do cotidiano. Esse projeto também irá contribuir para os alunos irem se familiarizando com projetos de engenharia, propondo uma interação entras as diversas áreas da engenharia e trabalho em grupo. 

 


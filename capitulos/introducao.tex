\chapter[Introdução]{Introdução}

\section{Contextualização}

\par Atualmente, no Brasil, o número de cidadãos com algum tipo de deficiência chega aos 23.9\% da população total, sendo 7\% apenas pessoas com algum tipo de deficiência motora \cite{ibge2010}. Assim, é necessário criar formas de incluir tais pessoas, que uma vez inclusas podem ser uma grande parte da força econômica de um país. 

Pensando nisso, foram criadas leis para que ambientes públicos se adequassem para eliminar barreiras arquitetônicas às pessoas com deficiência como as leis 10.048/2000 e 10.098/2000 \cite{abnt2004}. Entretanto, isso não é suficiente para acabar com o problema de acessibilidade; é apenas um passo para a inclusão total desses indivíduos. 

Nesse contexto, torna-se necessária a criação de soluções criativas para que barreiras cada vez maiores sejam derrubadas. Um problema que muitos usuários de cadeira de rodas enfrentam é a falta de independência em ações básicas como fazer compras em um supermercado.

Por vezes, o cadeirante depende de um acompanhante ou de se deslocar-se para outro veículo de propriedade do supermercado como \textit{scooters} para que possa realizar suas compras. O presente projeto busca sanar problemas relacionados à locomoção de cadeirantes neste ambiente.

\section{Termo de abertura do projeto}
De modo a abordar os principais tópicos do projeto, foi elaborado de forma resumida o Termo de Abertura do Projeto como propõe o \citeauthor{pmbok2013guia} (\citeyear{pmbok2013guia}). Esse termo foi preparado considerando o exemplo do \citeauthor{lappis_tap} (\citeyear{lappis_tap}).

\subsection{Descrição do projeto}

A solução visa diminuir as dificuldades de cadeirantes ao realizar suas compras em supermercados. A proposta consiste em desenvolver um carrinho de compras autônomo que possa "seguir" a pessoa com mobilidade reduzida. Assim, enquanto o cadeirante se desloca pelos corredores do supermercado, o carrinho será capaz de acompanhá-lo.

Será levado em consideração o design da estrutura do carrinho. Será dado enfoque em dimensões apropriadas para cadeirantes, levando em consideração a distância que o carrinho de supermercado ficará do cadeirante quando este efetuar uma parada para colocar produtos nos compartimentos do carrinho.

Dessa forma, esse carrinho deverá agir de forma autônoma e desviar de obstáculos em um ambiente ideal, ou seja, sem rampas. E assim facilitar e tornar mais ágil o momento de compras, de forma a oferecer uma certa independência ao cadeirante.

O projeto está dividio em três subsistemas principais:

\begin{itemize}
    \item \textbf{Alimentação}: composto por motors elétricos, carregador elétrico e baterias. O subsistema está detalhado na seção \ref{sec:alimentacao}).
    \item \textbf{Estrutura}: composto pelas rodas, transmissão e demais componentes estruturais do carrinho. O subsistema está detalhado na seção detalhado na seção \ref{sec:estrutura}.
    \item \textbf{Controle}: composto por hadware, sensoriamento e algoritmos de controle. O subsistema está detalhado na seção (\ref{sec:controle}).
\end{itemize}

\subsection{Objetivo geral}

O objetivo geral do projeto consiste em realizar a construção de um carrinho de compras de supermercado autônomo, ou seja, sem que haja controle pelo usuário.

\subsection{Objetivos específicos}

O projeto vigente tem como finalidades específicas:

\begin{itemize}  
\item Obter o produto com características ideais que cumpra os requisitos propostos para confecção do projeto;
\item Realizar o projeto com custos reduzidos;
\item Desenvolver algoritmo capaz de acompanhar o cadeirante;
\item Incorporar sensores para a detecção de possíveis obstáculos;
\item Estabelecer uma nova ergonomia para o carrinho de compras;
\item Incorporar ao produto elementos de Engenharia Eletrônica, de Software, Automotiva, de Energia e Aeroespacial;
\item Integrar dispositivos tendo suas responsabilidades divididas por áreas das engenharias.
\end{itemize}


\subsection{Escopo do projeto}

O escopo do projeto e o que está fora do mesmo é apresentado na Tabela \ref{tab:escopo}

% ######## init table ########
\begin{table}[h]
 \centering
% distancia entre a linha e o texto
 {\renewcommand\arraystretch{1.25}
 \caption{Escopo do projeto}
 \label{tab:escopo}
 \begin{tabular}{ l l }
  \cline{1-1}\cline{2-2}  
    \multicolumn{1}{c|}{\textbf{É escopo} \centering } &
    \multicolumn{1}{c}{\textbf{Não é escopo} \centering }
  \\  
  \cline{1-1}\cline{2-2}  
    \multicolumn{1}{p{3.850cm}|}{\begin{center}Que o carrinho seja autônomo
\end{center}  			


\begin{center}Que o carrinho desvie de obstáculos em curvas
\end{center}} &
    \multicolumn{1}{p{4.217cm}}{\begin{center}Que o carrinho funcione em rampas
\end{center}  			


\begin{center}Que o cadeirante controle o carrinho manualmente
\end{center}}
  \\  
  \hline

 \end{tabular} }
\end{table}


\pagebreak

\subsection{\textit{Stakeholders}}

Os \textit{stakeholders} e suas respectivas responsabilidades e expectativas são evidenciadas na Tabela \ref{tab:stakeholders}
% ######## init table ########
\begin{table}[h]
 \centering
% distancia entre a linha e o texto
 {\renewcommand\arraystretch{1.25}
 \caption{Stakeholders.}
 \label{tab:stakeholders}
 \begin{tabular}{ l l }
  \cline{1-1}\cline{2-2}  
    \multicolumn{1}{p{3.317cm}|}{\textit{Stakeholders}} &
    \multicolumn{1}{p{9.167cm}}{Responsabilidade/Expectativa}
  \\  
  \cline{1-1}\cline{2-2}  
    \multicolumn{1}{p{3.317cm}|}{Alunos} &
    \multicolumn{1}{p{9.167cm}}{Sâo responsáveis por realizar o projeto, construção e teste do carrinho.}
  \\  
  \cline{1-1}\cline{2-2}  
    \multicolumn{1}{p{3.317cm}|}{Pessoas com mobilidade reduzida (cadeirantes)} &
    \multicolumn{1}{p{9.167cm}}{Possui a expectativa de que o carrinho possa o auxiliar durante compras no supermercado de modo ergonômico e eficiente.}
  \\  
  \cline{1-1}\cline{2-2}  
    \multicolumn{1}{p{3.317cm}|}{Supermercado} &
    \multicolumn{1}{p{9.167cm}}{Possui a expectativa de adquirir o carrinho como possível vantagem competitiva com relação à concorrência.}
  \\  
  \hline

 \end{tabular} }
\end{table}


\subsection{Marcos do projeto}

O projeto possui três marcos principais orientados às datas dos pontos de controle da disciplina (Tabela \ref{tab:marcos}).

% ######## init table ########
\begin{table}[h]
 \centering
% distancia entre a linha e o texto
 {\renewcommand\arraystretch{1.25}
 \caption{Marcos do projeto.}
 \label{tab:marcos}
 \begin{tabular}{ l l l }
  \cline{1-1}\cline{2-2}\cline{3-3}  
    \multicolumn{1}{p{1.583cm}|}{Marco} &
    \multicolumn{1}{p{1.917cm}|}{Data} &
    \multicolumn{1}{p{8.950cm}}{Descrição}
  \\  
  \cline{1-1}\cline{2-2}\cline{3-3}  
    \multicolumn{1}{p{1.583cm}|}{1} &
    \multicolumn{1}{p{1.917cm}|}{02/09/2016} &
    \multicolumn{1}{p{8.950cm}}{Proposta e aprovação do projeto}
  \\  
  \cline{1-1}\cline{2-2}\cline{3-3}  
    \multicolumn{1}{p{1.583cm}|}{2} &
    \multicolumn{1}{p{1.917cm}|}{09/11/2016} &
    \multicolumn{1}{p{8.950cm}}{Protótipo de cada subsistema funcionando separadamente}
  \\  
  \cline{1-1}\cline{2-2}\cline{3-3}  
    \multicolumn{1}{p{1.583cm}|}{3} &
    \multicolumn{1}{p{1.917cm}|}{02/12/2016} &
    \multicolumn{1}{p{8.950cm}}{Integração dos subsistemas}
  \\  
  \hline

 \end{tabular} }
\end{table}

\subsection{Orçamento preliminar}

O orçamento inicial para o projeto considera o valor estimado de cada substistema bem como o custo com o trabalho de cada aluno \footnote{O valor de R\$16,50 foi obtido através do Relatório de Gestão do Exercício de 2015 \cite{unb2015}}. 
%tabela
% ######## init table ########
\begin{table}[h]
 \centering
% distancia entre a linha e o texto
 {\renewcommand\arraystretch{1.25}
 \begin{tabular}{ l l l l }
  \cline{1-1}\cline{2-2}\cline{3-3}\cline{4-4}  
    \multicolumn{1}{p{2.367cm}|}{\begin{center}\textbf{Recurso}
\end{center}} &
    \multicolumn{1}{p{2.367cm}|}{\begin{center}\textbf{Valor estimado}
\end{center}} &
    \multicolumn{1}{p{2.367cm}|}{\begin{center}\textbf{Quantidade}
\end{center}} &
    \multicolumn{1}{p{2.367cm}}{\begin{center}\textbf{Duração}
\end{center}}
  \\  
  \cline{1-1}\cline{2-2}\cline{3-3}\cline{4-4}  
    \multicolumn{1}{p{2.367cm}|}{\begin{center}Engenheiro
\end{center}} &
    \multicolumn{1}{p{2.367cm}|}{\begin{center}R\$ 16,50 por hora
\end{center}} &
    \multicolumn{1}{p{2.367cm}|}{\begin{center}13
\end{center}} &
    \multicolumn{1}{p{2.367cm}}{\begin{center}90h
\end{center}}
  \\  
  \cline{1-1}\cline{2-2}\cline{3-3}\cline{4-4}  
    \multicolumn{1}{p{2.367cm}|}{\begin{center}Subsistema Controle
\end{center}} &
    \multicolumn{1}{p{2.367cm}|}{\begin{center}R\$ 665,00
\end{center}} &
    \multicolumn{1}{p{2.367cm}|}{\begin{center}1
\end{center}} &
    \multicolumn{1}{p{2.367cm}}{\begin{center}90h
\end{center}}
  \\  
  \cline{1-1}\cline{2-2}\cline{3-3}\cline{4-4}  
    \multicolumn{1}{p{2.367cm}|}{\begin{center}Subsistema Estrutura
\end{center}} &
    \multicolumn{1}{p{2.367cm}|}{\begin{center}R\$ 300,00
\end{center}} &
    \multicolumn{1}{p{2.367cm}|}{\begin{center}1
\end{center}} &
    \multicolumn{1}{p{2.367cm}}{\begin{center}90h
\end{center}}
  \\  
  \cline{1-1}\cline{2-2}\cline{3-3}\cline{4-4}  
    \multicolumn{1}{p{2.367cm}|}{\begin{center}Subsistema Alimentação
\end{center}} &
    \multicolumn{1}{p{2.367cm}|}{\begin{center}R\$ 800,00
\end{center}} &
    \multicolumn{1}{p{2.367cm}|}{\begin{center}1
\end{center}} &
    \multicolumn{1}{p{2.367cm}}{\begin{center}90h
\end{center}}
  \\  
  \cline{1-1}\cline{2-2}\cline{3-3}\cline{4-4}  
    \multicolumn{1}{p{2.367cm}}{\begin{center}Total
\end{center}} &
    \multicolumn{3}{p{7.100cm}}{\begin{center}R\$ 3.250,00
\end{center}}
  \\  
  \hline

 \end{tabular} }
\end{table}


\subsection{Gerenciamento do projeto}

O gerenciamento do projeto é realizado através de planos de gerenciamento de tempo, de riscos, de recursos humanos, de custos, de tempo e de requisitos. Os planos podem ser encontrados no Apêndice \ref{appendix:planos}.


\section{Contribuição das engenharias}

A solução irá cobrir conceitos de cada engenharia participante do projeto, através da troca e aplicação de conhecimentos resultando em subsistemas do produto final. 


Quanto às contribuições realizadas pela Engenharia de Energia, destacam-se:

\begin{itemize}
    \item Dimensionamentos dos motores e das baterias de modo a atender aos requisitos do carrinho, tais como o peso que este irá suportar e a respectiva potência que precisará; bem como o torque de partida do motor;
	\item Auxílio na confecção da estrutura, na própria usinagem e também em pesquisas quanto aos materiais que podem ser utilizados; 
	\item Desenvolvimento do carregador da bateria com a ajuda dos membros da Engenharia Eletrônica.
\end{itemize}

Quanto às contribuições da Engenharia Eletrônica, destacam-se:

\begin{itemize} 
    \item Localização do cadeirante, por meio de emissão/recepção de ondas eletromagnéticas
    \item Evitar todo e qualquer tipo de colisão a partir de algoritmos de controle de motor DC
    \item Projeto e fabricação de placas de circuito impresso de comunicação em Radio-frequência e Ponte H;
    \item Especificação de sensores de distância adequados.
\end{itemize}
 
Quanto às contribuições de Engenharia de Software, destacam-se:

\begin{itemize}
    \item Algoritmo de processamento de imagens em tempo real;
    \item Algoritmo de captura da distância através do sensor ultrassom;
    \item Algoritmo de controle dos motores integrado aos algoritmos supracitados;
    \item Aplicação da técnica PID para selecionar os melhores \textit{inputs} de distância.
    \item Utilização de questionário como técnica de identificação de requisitos;
    \item Especificação e documentação: definição de requisitos funcionais e não-funcionais;
    \item Representação gráfica dos algoritmos;
    \item Gerência de configuração de software:  identificação da configuração dos algoritmos com a finalidade de controlar mudanças e configurar o ambiente de desenvolvimento e de implantação.
\end{itemize}

Quanto às contribuições da Engenharia Automotiva, destacam-se:

\begin{itemize}
    \item Realização do projeto de estrutura e análise estrutural de forma analítica e numérica, verificando também a parte de transmissão de potência do motor para as rodinhas, de modo que o carrinho consiga se movimentar considerando o seu peso máximo;
    \item Fabricação da estrutura e de suas peças componentes, determinando o melhor material a ser utilizado, para garantir a resistência estrutural, o melhor processo de fabricação a ser utilizado e garantindo a integração de todos os componentes.
\end{itemize}

Quanto às contribuições da Engenharia Aeroespacial, destacam-se:

\begin{itemize}
    \item Auxiliar no projeto e construção da estrutura do carrinho;
    \item Realizar análise numérica bem como otimização da geometria e escolhendo melhor material a ser utilizado.
\end{itemize}





